\chapter{Substitution and Beta Reduction}
\label{ch:lambda}

\begin{preamble}
\href{ch:lcs}{Lambda Calculus} allows write code where function abstraction (lambda terms) and function application are the only means of computation.
%
In this chapter, we will see how such computation may be performed by using substitution and beta reduction. 
\end{preamble}

\section{Substitution}
\label{sec:lcsb::sub}

In traditional mathematics, the fundamental operation by which human beings ``calculate'' or ``compute'' is via substitution.
%
Given a function with some parameter and an argument for the function, we plug in the argument for the parameter.
%
For example, if the function is written as 
\[
f(x) = x + 1,
\]  
then we calculate $f(5)$ by substituting $5$ for $x$ in the ``body'' of $f(\cdot)$, which we may write as $[5/x](x+1)$, to obtain $6$.
%
We can define substitution for lambda calculus in essentially the same
way.

\begin{definition}[Substitution]
\label{def:lcsb::basic}
The \defn{substitution} of a term, $t'$ for a variable $x \in V$ in a term
$t$, denoted by $[t'/x]t$, is an instance of $t$ where all the free
occurrences $x$ is replaced by the term $t'$.
\end{definition}

Throughout this course, we will use the notation $[t'/x]t$ to denote a
substitution of $t'$ for $x$ in $t$.  Different notations are
preferred by different textbooks or authors. Other commonly used
notations include $t[t'/x]$, $[x:=t']t$, and  $[x \leftarrow t']t$.  

\begin{example}
\label{xmpl:lcsb::basic}
Some example substitutions follow
\begin{enumerate}
\item $[\l y.y/x] (x~x)= (\l y.y)~(\l y.y)$

\item $[\l y.y/x] (\l x.x) = \l x.x$ ($x$ is not free.)

\item $[y/x](\l z.x) = \l z.y$

\end{enumerate}
\end{example}

\begin{gram}[Capture]
\label{grm:lcsb::capture}
A substitution, as we defined it, is actually incorrect.  
%
For example, consider the lambda abstraction $\l y.x$, the constant
function, and the substitution $[y/x](\l y.x)$, which is equal to $\l
y.y$, the identity function.
%
Thus, this substitution changes the constant function into the
identity function.
%
The problem is that the variable $x$, a free
variable, is substituted with $y$, which then becomes bound by the
lambda abstraction.  In this case, we say that $y$ is \defn{captured}
in the substitution.
%
We can define the ``capture-avoiding substitution'' as follows. 
\end{gram}


\begin{definition}[Capture-Avoiding substitution]
\label{def:lcsb::sub-avoid}
\[
[t/x]y = \left\{ \begin{array}{ll}
        t & if~y = x \\
        y & if~y\neq x \\
        \end{array} \right.
\]


\[
[t/x](t_1~t_2) = [t/x]t_1~[t/x]t_2.
 \]

\[
[t'/x](\l y.t) = 
\left\{ \begin{array}{ll}

\l y.t &\mbox{if}~ x = y \mbox{ (y is bound)} \\

\l y.[t'/x]t &\mbox{if}~ x \neq y \mbox { and } y \not \in FV(t') \\

\end{array} \right. 
\]
\end{definition}

\begin{flex}
\begin{exercise}
\label{xrcs:lcsb::sub-undef}
Using Definition \ref{def:lcsb::sub-avoid},
%
what does the substitution $[\l y.x~y~y/z](\l x.x~z)$ yield?
\end{exercise}
\begin{solution}
\label{sol:lcsb::sub-basic}
This substitution is undefined.
\end{solution}
\end{flex}

\begin{flex}
\begin{exercise}
\label{xrcs:lcsb::sub-rename}
Using \ref{def:lcsb::sub-avoid},
How about the substitution  $[\l y.x~y~y/z](\l y.y~z)$
\end{exercise}
\begin{solution}
$\l y.y~(\l y.x~y~y)$
\end{solution}
\end{flex}



We do not want to transform free variables into bound variables during
the substitution process. 

 We need to redefine substitution to avoid
capture. What can we do?  Consider the following situation: given a
program which has a function that takes a variable $x$ and has a
variable $y$ in its body, we can change the name of $y$ to $z$ without
any modification of the results, but we cannot change the name of $y$
to $x$ without adversely affecting the results. The next definition
avoids capture.
 



According to this definition a substitution that causes capture is
undefined.  But note that in the two exercises above, the lambda
abstractions $\l x.x~z$ and $\l y.y~z$ differ only in the name of
their bound variables.  
%
Lambda abstractions denote functions, thus
these two lambda abstractions are the same; the names of the
the bound variables should not matter.
%
Indeed, one way to avoid capture is to rename the bound variables in a
lambda abstraction.  To goal is to ensure that no bound variable has
the same name as a free variable in the term being substituted.


\begin{definition}[Substitution with Explicit Alpha Conversion]
\label{def:lcsb::exp}
\[
[t/x]y = \left\{ \begin{array}{ll}
        t & if~y = x \\
        y & if~y\neq x \\
        \end{array} \right.
\]


\[
[t/x](t_1~t_2) = [t/x]t_1~[t/x]t_2.
 \]



\[[t'/x](\l y.t)  =
\left\{ \begin{array}{ll}
        \l y.t&\mbox{if}~x = y \\
        \l z. [t'/x] [z/y] t
        &\mbox{if}~ x \not= y ~ \land ~ z \not\in FV(t) \cup FV(t') \\
       \end{array} \right. \]
\end{definition}

The idea of this definition is to rename formal variable of the lambda
abstraction so that it does not occur freely in $t'$.  This ensures
that no free variables of $t'$ are captured.  This is not enough,
however, because we also have to make sure that we do not capture the
free variables of $t$ itself. 

Note that, in this definition, substitution is a relation not a
function.  That is the result of a substitution is a set of terms
where the renaming take different forms.  More precisely, the variable
$z$ can take many values (the names of difference variables).  All
such terms are $\alpha$-equivalent. 

Now that we made precise the idea of alpha conversion, we can now
forget about it.  From now on, we will work modulo
$\alpha$-equivalence.  That is, we will not distinguish between two
terms that are $\alpha$-equivalent.  We can use our old definition,
with implicit alpha conversion applied as required.


\begin{definition}[Substitution with Implicit Alpha Conversion]
\label{def:lambda:sub-final}
\[
[t/x]y = \left\{ \begin{array}{ll}
        t & if~y = x \\
        y & if~y\neq x \\
        \end{array} \right.
\]


\[
[t/x](t_1~t_2) = [t/x]t_1~[t/x]t_2.
 \]

\[
[t'/x](\l y.t) = 
\left\{ \begin{array}{ll}

\l y.t &\mbox{if}~ x = y \mbox{ (y is bound)} \\

\l y.[t'/x]t &\mbox{if}~ x \neq y \land y \not \in FV(t') \\

\end{array} \right. 
\]
\end{definition}

\section{$\beta$-reduction}

In \secref{substitution}, we briefly discussed how to evaluate a
function with a given parameter by substitution.  We can give an
operational semantics for lambda terms based on substitution.  The
idea is to take each application and reduce it to another term by
applying substitution---this is called \emph{$\beta$ reduction}, and is
denoted $\redb$.

\begin{definition}[Single-step $\beta$-reduction]
\[
\begin{array}{c}
(\l x.t_1)~t_2 \redb [t_2/x] t_1 \\[4mm]
\infer{t_1~t_2 \redb t_1'~t_2}{t_1 \redb t'_1}\\[4mm]
\infer{t_1~t_2 \redb t_1~t'_2}{t_2 \redb t'_2}\\[4mm]
\infer{\l x.t \redb \l x.t'} {t \redb t'}
\end{array}
\]
\end{definition}


We define multi-step beta reduction, denoted $\redbs$, as $0$ or more
applications of single-step beta reduction rules.  The use of $*$ to
denote reductions suggests their use in Kleene Algebra of Automata
Theory.

\begin{definition}[Multi-step $\beta$-reduction]
\[
\begin{array}{c}
\infer{\strut} {t \redbs t}\\[4mm]
\infer{t \redbs t'} {t \redb t'} \\[4mm]
\infer{t_1 \redbs t_3} {t_1 \redbs t_2 \\t_2 \redbs t_3} \\[4mm]
\end{array}
\]
\end{definition}

Two terms that are $\beta$-reducible to each other are called
$\beta$-equivalent, denoted by $\eqb$. 

\begin{definition}[$\beta$-equivalence]
\[
\begin{array}{c}
\infer{\strut} {t \eqb t} \\[4mm]
\infer{t \eqb t'} {t \redb t'}  \\[4mm]
\infer{t_1 \eqb t_3} {t_1 \eqb t_2 \\t_2 \eqb t_3}  \\[4mm]
\infer{t' \eqb t} {t \eqb t'}
\end{array}
\]
\end{definition}


\subsection{Evaluation Strategies}

The only means of computing the lambda calculus is beta reduction,
i.e., the application of a function to its arguments.  In beta
reduction, the expression being reduced is called a \emph{reducible
  expression or a redex}.  Given a term, we evaluate that term by
applying beta reduction to its redexes.  There a often multiple
redexes in a term, and, in general, we can choose to evaluate any one
of them.  The following \emph{evaluation strategies} have been
proposed.  As we will see in the rest of the class, there are
interesting differences between these strategies in terms of
``run-time'' behavior of programs.

\begin{description}
\item[Full beta reduction:] Any redex can be reduced at any time.
\item[Normal order strategy:] The leftmost and outermost redex is
  reduced next.
\item[Call by name strategy:] The leftmost and outermost redex is
  reduced next but no reductions are allowed under abstractions.
\item[Call by value:] Like call by name but a redex is reduced only
  when its right hand-side is a value.
\end{description}


Here is a call-by-value operational semantics for lambda calculus. 

\[
\begin{array}{c}
\infer{t_1~t_1 \red t_1'~t_2}{t_1 \red t_1'}\\[2mm]
\infer{v_1~t_2 \red v_1~t_2'}{t_2 \red t_2'}\\
\infer{(\l x.t) v \red [v/x]~t}{\strut}
\end{array}
\]

\section{Homework Exercise}
A term $t$ is in normal form if there is no $t'$ such that $t \redb
t'$.  We say that a term $t$ is \emph{normalizable} if there is some
$t'$ such that $t \redbs t'$ and $t'$ is in normal form.  Answer the
following questions and prove your answer (if your answer is yes, an
example suffices).

\begin{enumerate}
\item Are there any normalizable terms?
\item Are there any non-normalizable terms?
\end{enumerate}







