\chapter{Lambda Calculus: Beta Reduction}
\label{ch:lambda}

\begin{preamble}
This chapter presents the ``beta reduction'', a technique for ``running'' or evaluating \href{ch:lcs}{Lambda Calculus} expressions.
%
Beta reduction enables computing with Lambda Calculus.
\end{preamble}

\section{Substitution}
\label{sec:lcsb::sub}

In traditional mathematics, the fundamental operation by which human beings ``calculate'' or ``compute'' is via substitution.
%
Given a function with some parameter and an argument for the function, we plug in the argument for the parameter.
%
For example, if the function is written as 
\[
f(x) = x + 1,
\]  
then we calculate $f(5)$ by substituting $5$ for $x$ in the ``body'' of $f(\cdot)$, which we may write as $[5/x](x+1)$, to obtain $6$.
%
We can define substitution for lambda calculus in essentially the same
way.

\begin{definition}[Substitution]
\label{def:lcsb::basic}
The \defn{substitution} of a term, $t'$ for a variable $x \in V$ in a term
$t$, denoted by $[t'/x]~t$, is an instance of $t$ where all the free
occurrences $x$ is replaced by the term $t'$.
\end{definition}

Throughout this course, we will use the notation $[t'/x]t$ to denote a
substitution of $t'$ for $x$ in $t$.  Different notations are
preferred by different textbooks or authors. Other commonly used
notations include $t[t'/x]$, $[x:=t']t$, and  $[x \leftarrow t']t$.  

\begin{example}
\label{xmpl:lcsb::basic}
Some example substitutions follow
\begin{enumerate}
\item $[\l y.y/x]~(x~x)= (\l y.y)~(\l y.y)$

\item $[\l y.y/x]~(\l x.x) = \l x.x$ ($x$ is not free.)

\item $[y/x](\l z.x) = \l z.y$

\end{enumerate}
\end{example}

\begin{gram}[Capture]
\label{grm:lcsb::capture}
A substitution, as we defined it, is actually incorrect.  
%
For example, consider the lambda abstraction $\l y.x$, the constant
function, and the substitution $[y/x](\l y.x)$, which is equal to $\l
y.y$, the identity function.
%
Thus, this substitution changes the constant function into the
identity function.
%
The problem is that the variable $x$, a free
variable, is substituted by $y$, which then becomes bound by the
lambda abstraction.  In this case, we say that $y$ is \defn{captured}
by the substitution.
%
We can define the ``capture-avoiding substitution'' as follows. 
\end{gram}


\begin{definition}[Capture-Avoiding substitution]
\label{def:lcsb::sub-avoid}
\[
[t/x]y = \left\{ \begin{array}{ll}
        t & if~y = x \\
        y & if~y\neq x \\
        \end{array} \right.
\]


\[
[t/x](t_1~t_2) = [t/x]t_1~[t/x]t_2.
 \]

\[
[t'/x](\l y.t) = 
\left\{ \begin{array}{ll}

\l y.t &\mbox{if}~ x = y \mbox{ (y is bound)} \\

\l y.[t'/x]t &\mbox{if}~ x \neq y \mbox { and } y \not \in FV(t') \\

\end{array} \right. 
\]
\end{definition}

\begin{flex}
\begin{exercise}
\label{xrcs:lcsb::sub-undef}
Using Definition \ref{def:lcsb::sub-avoid},
%
what does the substitution $[\l y.x~y~y/z](\l x.x~z)$ yield?
\end{exercise}
\begin{solution}
\label{sol:lcsb::sub-basic}
This substitution is undefined.
\end{solution}
\end{flex}

\begin{flex}
\begin{exercise}
\label{xrcs:lcsb::sub-rename}
Using Definition \ref{def:lcsb::sub-avoid},
what does the substitution  $[\l y.x~y~y/z]~(\l y.y~z)$ yield?
\end{exercise}
\begin{solution}
$\l y.y~(\l y.x~y~y)$
\end{solution}
\end{flex}


We can refine
%
Definition \ref{def:lcsb::sub-avoid},
%
to deal with capture by actively renaming bound variables.
%
The idea is hinted by the two exercises above:  the lambda
abstractions $\l x.x~z$ and $\l y.y~z$ used in the examples differ only in the name of their bound variables.  
%
We know that these two abstractions are 
%
\href{def:lcs::alpha}{alpha-equivalent}, and
%
we can alpha-convert lambda abstractions to avoid capture.  

\begin{flex}
\begin{definition}[Substitution with Explicit Alpha Conversion]
\label{def:lcsb::sub-exp}
\[
[t/x]~y = \left\{ \begin{array}{ll}
        t & if~y = x \\
        y & if~y\neq x \\
        \end{array} \right.
\]


\[
[t/x]~(t_1~t_2) = \left([t/x]~t_1\right)~\left([t/x]~t_2\right).
 \]



\[
[t'/x]~(\l y.t)  =
\left\{ \begin{array}{ll}
        \l y.t&\mbox{if}~x = y \\
        \l z. [t'/x]~[z/y]~t
        &\mbox{if}~ x \not= y ~ \land ~ z \not\in FV(t) \cup FV(t') \\
       \end{array} \right. \]
\end{definition}

\begin{remark}
The idea of this definition is to alpha convert lambda
abstractions so that its formal parameter does not occur freely in $t'$.
%
When renaming the formal parameter, it is important to choose a variable that does not occur freely in the body of the lambda term and in the term being substituted.
\end{remark}

\begin{note}
The substitution above is a relation and not a function: the result of
a substitution is a set of terms that differ based on the choice of
the variable name chosen.  All such terms are $\alpha$-equivalent.
\end{note}
\end{flex}

\begin{gram}[Working Modulo Alpha Conversion]
Now that we made precise the idea of alpha conversion and how
substitution via alpha conversion works, we can now forget about it.
%
From now on, we will work modulo $\alpha$-equivalence.  That is, we
will not distinguish between two terms that are $\alpha$-equivalent.
%
We can  use a simpler  definition for substitution, with implicit alpha conversion applied as needed; this definition is reproduced below for convenience.
\end{gram}


\begin{definition}[Substitution with Implicit Alpha Conversion]
\label{def:lambda:sub-final}
\[
[t/x]y = \left\{ \begin{array}{ll}
        t & if~y = x \\
        y & if~y\neq x \\
        \end{array} \right.
\]


\[
[t/x](t_1~t_2) = [t/x]t_1~[t/x]t_2.
 \]

\[
[t'/x](\l y.t) = 
\left\{ \begin{array}{ll}

\l y.t &\mbox{if}~ x = y \mbox{ (y is bound)} \\

\l y.[t'/x]t &\mbox{if}~ x \neq y \land y \not \in FV(t') \\

\end{array} \right. 
\]
\end{definition}

\section{Beta Reduction}

Lambda calculus has a remarkably simple syntax.  It claims that all you need to compute is a variable,  function abstraction, and function application.
%
The actual computation itself is also equally elegant: all we need is a single rule: beta reduction, which ``reduces'' function applications via substitution.
%
The ultimate goal of beta reduction is to reduce to a term to a term that cannot be evaluated further, i.e., a normal form.

\begin{definition}[Normal Form]
\label{def:lcb::normal-form}
A term $t$ is in  \defn{normal form} if there is no $t'$ such that $t \redb
t'$.  
\end{definition}


\begin{definition}[Reducible Expression and Beta Reduction]
\label{def:lcb::beta-reduction}
We say that a term of the form 
\[
(\l x.t_1)~t_2 
\]
is  a \defn{reducible expression} or \defn{redex},
%
and
%
define \defn{beta reduction} as a function that maps a redex to its substituted form, i.e.,
\[
(\l x.t_1)~t_2 \redb [t_2/x]~t_1 .
\]
\end{definition}


\begin{definition}[$\beta$-equivalence]
\label{def:lcb::beta-eq}
Two terms that are $\beta$-reducible to each other are called
\defn{$\beta$-equivalent}, denoted by $\eqb$. 
\[
\begin{array}{c}
\infer {t \eqb t} {\strut} 
\\[4mm]
\infer {t \eqb t'} {t \redb t'}  
\\[4mm]
\infer {t_1 \eqb t_3} {t_1 \eqb t_2 \\ t_2 \eqb t_3}  
\\[4mm]
\infer {t' \eqb t} {t \eqb t'}
\end{array}
\]
\end{definition}


\begin{example}
\label{xmpl:lcb::beta-reduction-1}
The following term is a redex
\[
(\l z. z~(\l x. \l y. x))~(\l y. y~x_1~x_2), 
\]
and thus can be reduced via beta reduction
\[
(\l z. z~ (\l x. \l y. x))~(\l y. y~x_1~x_2) 
\redb
(\l y. y~x_1~x_2)~(\l x. \l y. x).
\]

Now the resulting term is also a redex and can be reduced, and in fact a few more steps:
\[
(\l y. y~x_1~x_2)~(\l x. \l y. x).
\redb
(\l x. \l y. x)~~x_1~x_2
\redb
x_1.
\]
\end{example}


Reducing an expression to normal form typically requires applying beta reduction multiple times.
% 
We define multi-step beta reduction, denoted $\redbs$, as $0$ or more
applications of single-step beta reduction rules.  The use of $*$ to
denote reductions is borrowed from the work of 
%
\href{https://en.wikipedia.org/wiki/Stephen_Cole_Kleene}{Kleene},
%
one of the founders of computability theory.

\begin{definition}[Multi-step $\beta$-reduction]
\label{lcb:beta::betastar}
\[
\begin{array}{c}
\infer {t \redbs t} {\strut} \\[4mm]
\infer {t \redbs t'} {t \redb t'}  \\[4mm]
\infer {t_1 \redbs t_3} {t_1 \redbs t_2 \\t_2 \redbs t_3} \\[4mm]
\end{array}
\]
\end{definition}



\begin{definition}[Normalizable Terms]
\label{def:lcb::normalizable}
We say that a term $t$ is \defn{normalizable} if there is some
$t'$ such that $t \redbs t'$ and $t'$ is in normal form. 
\end{definition}


\section{Evaluation Strategies}

Beta reduction allows evaluating a lambda term by repeatedly performing substitution.  Because there can be many redexes in a lambda term, there can be many ways to evaluate a term.  
%
Over the many decades, researchers have proposed various evaluation algorithms or strategies, including the following.
%
\begin{description}
\item[Full beta reduction:] Any redex can be reduced at any time.
\item[Normal order strategy:] The leftmost and outermost redex is
  reduced next.
\item[Call by name strategy:] The leftmost and outermost redex is
  reduced next but no reductions are allowed under abstractions.
\item[Call by value:] Like call by name but a redex is reduced only
  when its right hand-side is a value.
\end{description}


\begin{gram}[Specifying Call-by-Value]

We can specify the call-by-value algorithm by using inference rules:
\[
\begin{array}{c}
\infer {(\l x.t) v \red [v/x]~t} {\strut}
\\[2mm]
\infer {t_1~t_1 \red t_1'~t_2} {t_1 \red t_1'}
\\[2mm]
\infer {v_1~t_2 \red v_1~t_2'} {t_2 \red t_2'}
\end{array}
\]
\end{gram}

\begin{exercise}
Specify the full-beta-reduction algorithm using inference rules.
\end{exercise}



\begin{exercise}
Are there terms in lambda calculus that are normalizable?  Are there terms that are not normalizable?
\end{exercise}

\begin{exercise}
Show that call-by-name and call-by-value algorithms are deterministic. 
%
That is for each, show that if $t \red t'$ and $t \red t''$, then $t' =
t''$.
\end{exercise}


%% \begin{solution}

%% \begin{enumerate}
%% \item Are there any normalizable terms?  

%%   Yes there are many normalizable terms.  Variables are in normal
%%   form, because we cannot reduce them further, but they are not that
%%   interesting.  The interesting terms are closed terms that we cannot
%%   reduce further.  For example, the term $\l~x.x$.  In general, if you
%%   define a language and a set of values that are consistent with its
%%   operational semantics, then any value will be in normal form.

%% \item Are there any non-normalizable terms?

%%   Yes, there are many non-normalizable terms.  In general, the terms
%%   that ``diverge'' or reproduce themselves are not in normal form.
%%   For example, the term $(\l~x.x~x)~(\l~x.x~x)$ diverges, because it
%%   beta-reduces to itself.

%% \end{enumerate}
%% \end{solution}






