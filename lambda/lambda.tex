\chapter{ Lambda Calculus}
\label{ch:lambda}


\section{Some History}
\label{sec:lambda::history}

\begin{gram}
\label{grm:lambda::history}
In 1928, Hilbert and Ackerman posed a challenge: devise an algorithm that takes as input a  first-order logic statement and determines whether that statement is valid or not.
%
Soon after, Alonzo Church, then Professor at the Department of Mathematics in Princeton, 
started working on this problem.  His approach was to research the notion of ``function'' and create based on this notion a logical system that is sufficient for all of mathematics.
%
Lambda calculus emerged out of this research, also with contributions from 
Church's students Kleene and Rosser.
%
This research led to Church's 1936 paper 
\href{http://www.umut-acar.org/greats/church-1936.pdf}{showing that an algorithm as desired Hilbert and Ackerman's does not exist}.
%
His solution was to formulate a term in Lambda Calculus and show that there is no way to determine whether that term has a closed form (more precisely $\beta$-normal form). 
%
About one year later, Turing published \href{http://www.umut-acar.org/greats/turing-1937.pdf}{his paper}, where he establish the same result but using different techniques that are based on ``computing machines'', and 
%
proved that his and Church's approach were equivalent.

\end{gram}

\begin{gram}[Church and Turing]
\label{grm:lambda::church-and-turing}
Church and Turing's results are like two sides of a coin.  
%
Church's result is all about abstraction  offers a mathematical language in which computation can be expressed.
%
Turing's result is all about implementation: it convincingly describes how to implement computation.
%
\end{gram}

\section{Abstract Syntax of Lambda Calculus}
\label{sec:lambda::syn} 

There are at least several ways to define the syntax of Lambda Calculus.
%
In this section, we go through these different ways, partly because they introduce some basic techniques that we shall use in this course.  

\begin{definition}[Inductive Definition]
\label{def:lambda::syn::inductive}
Let $V$ be a countable set $V$ of variables.  We define the abstract
syntax for lambda calculus inductively as follows.
%
$\terms$ is the {\em least} set of the terms that satisfy the following.
\begin{enumerate}
\item{if $x\in V$ then $x\in {\cal T}$}
\item{if $t_1\in {\cal T}$ and $t_2\in {\cal T}$ then $t_1t_2\in {\cal T}$}
\item{if $x\in V$ and $t\in {\cal T}$ then $\l   x.t\in {\cal T}$}
\item{${\cal T}$ is the ``least'' set verifying the above properties}
\end{enumerate}

Each term in \terms is called a \defn{lambda term}.  
\end{definition}

\begin{example}
\label{xrcs:lambda::syn::1} 
Example lambda terms include
\begin{itemize}
\item $x$,
\item $\l x.x$, and
\item $\l x.x~y$.  
\end{itemize}
\end{example}

\begin{definition}[Lambda Abstraction and Application]
\label{def:lambda::syn::abstraction}
The term $\l x.x$ is called \defn{(lambda) abstraction},
%
and the term $t_1~t_2$ is known as \defn{ application}.
%

An intuitive way of thinking of $\l x.t$ is as a function that takes
$x$ and computes the result in its body $t$.
\end{definition}

\begin{gram}[Abstract versus Concrete Syntax]
\label{grm:lambda::syn::abs-and-conc} 
This inductive definition of lambda calculus is an \emph{abstract syntax}:
%
it defines the set of properly parsed terms (i.e., abstract syntax
trees).  
%

It does not tell us how to read a lambda term as written in
\emph{concrete syntax}.  
%
For example, given $\l x. x~y$, we can parse it as 
$(\l x.x~y)$ or $(\l x. x) y$.  
%
Similarly, we can parse $t_1~t_2~t_3$ as
$(t_1~t_2)~t_3$ or $t_1~(t_2~t_3)$.
\end{gram}

\begin{gram}[Disambiguation Conventions]
\label{grm:lambda::syn::conventions} 
We will use parenthesis to aid in parsing (to disambiguate the
syntax).  To minimize parenthesis, we will have the following
conventions:
\begin{enumerate}
\item Application associates to the left.
\item The body of a lambda terms extends as far as right as possible. 
\end{enumerate}

With this convention $\l x. x~y$ is parsed as $\l x. (x~y)$ and $t_1~
t_2~t_3$ is parsed as $(t_1~t_2)~t_3$.
\end{gram}


\begin{definition}[Inference Rules]
\label{def:lambda::syn::inference}
We can define the abstract syntax of Lambda Calculus by using inference rules.
%
Given a countable set of variables $V$, the set of lambda terms is
defined as follows.

\begin{enumerate}

\item{$\infer {x\in {\cal T}} {x\in V}$}

\item{$\infer  {t_1~t_2\in {\cal T}} {t_1\in {\cal T} \\t_2\in {\cal T}}$}

\item{$\infer  {\l x.t\in {\cal T}} {x\in V \\t\in {\cal T}}$}
\end{enumerate}

\end{definition}

\begin{exercise}
Convince yourself that the two definitions above, 
%
the  \href{def:lambda::syn::inductive} and \href{def:lambda::syn::inference}
%
are equivalent.
\end{exercise}

\begin{definition}[BNF Style]
\label{def:lambda::syn::bnf} 
We can define the syntax of Lambda Calculus by using the BNF style. 
%
Assuming that $x$ ranges over a countable set of variables, the set of
lambda terms $t$ is defined as follows.

\[
t \bnfdef x \bnfalt t_1 t_2 \bnfalt \l x.t
\]
\end{definition}



\begin{exercise}
\label{xrcs:lambda::syn::bnf} 
Suppose that our syntax allow us to write natural numbers and add
them.  For example, we may have terms like this $1, 2, +~1~2$.

Now, what does the following lambda terms do?
\begin{enumerate}
\item $\l x. +~x~1$,
\item $\l x. \l y. +~x~y$,
\item $\l x. \l y. \l z.  z (+~x~y)$.
\end{enumerate}
\end{exercise}


\begin{gram}[Summary]
\label{grm:lambda::syn::summary} 
What is remarkable about Lambda Calculus is that just a one line definition suffices to define all of computation.
%
We don't need to talk about ``tapes'', ``cells'' ``instructions'', ``states'', etc, which are needed even in the most basic definition of Turing Machine.
%
Not only it is  elegant, but it is also powerful: it allows expressing sophisticated algorithms clearly and concisely.
%

\end{gram}

\begin{note}
\label{nt:lambda::syn::pl} 
Even many modern programming languages today lack the features of Lambda Calculus.
%
Though many are also busy trying to add them (which is not always easy).
%
For example,
the C++17 has now support for Lambda expressions as outlined 
%
\href{http://www.open-std.org/jtc1/sc22/wg21/docs/papers/2016/p0170r1.pdf}
{in this document}
%
from the C++ Standard Committee.

For good measure, perhaps they could have waited a few more
years to match the 100th year invention of Lambda Calculus!
\end{note}


\section{Bound and Free Variables}
\label{sec:lambda::variables}

\begin{definition}[Argument and Bindig]
Consider a lambda abstraction of the form $\l x. t$, which denotes a
function.
%
We refer to the variable $x$ as the \defn{formal parameter} or more
simply as the \defn{argument} of the abstraction (function),
%
and
%
say that the abstractions \defn{binds} $x$.
%
\end{definition}

\begin{definition}[Bound and Free Variables]
The occurrence of a variable is \defn{bound} if there is an enclosing
lambda abstraction that binds the variable, and is \defn{free}
otherwise.
\end{definition}

\begin{example}
Consider the lambda abstraction, $\l x. x + y$.  In order to evaluate
the function, for a particular argument $x$, we need to know
the value of $y$.  The variable $y$ in this case is free and $x$
is bound; $\l$ binds $x$.  
\end{example}


\begin{example} 
  Some example lambda abstractions and their bound and free variables.
\begin{enumerate}
\item $\l x. x$ is the identity function. Here $x$ is a bound variable.
\item $\l x. y$ is the constant ``$y$'' function. Here $y$ is a free variable.
\item $\l x. x~y$ has $x$ as a bound variable and $y$ as a free variable.
\item $\l x. (\l y. x~y)$ has both variables bound.
\item $~x~\l x. x + 1$.  In this term the first occurrence of $x$ is
  free and the second is bound.
\end{enumerate}
\end{example}


\begin{definition}[Free Variables]
\label{def:FV}
The set of \defn{free variables} of a term $t$, written as $FV(t)$,
is defined as 
\begin{enumerate}
\item if $t = x$ then $FV(x) = \{ x \}$, or
\item if $t = t_1~t_2$ then $FV(t_1~t_2)=FV(t_1) \cup FV(t_2)$, and
\item if $t = \l x. t$ then $FV(\l x.t)=FV(t) \setminus \{x\}$
\end{enumerate}
\end{definition}

\begin{definition}[Closed and Open Terms]
A term $t$ is closed if $FV(t) = \varnothing$. Otherwise $t$ is open. 
\end{definition}




\section{Substitution}
\label{sec:substitution}

A lambda abstraction denotes a function.  How do we evaluate a
function or a lambda abstraction at a particular value?  For example,
the term $(\l x.+~1~x)2$ denotes the application of $(\l x.+~1~x)$ to
the parameter $2$.  To evaluate such a term we would like to replace
the occurrences of the formal parameter $x$ with the value $2$.  For
example, for $(\l x.+~1~x)~2$ evaluates to $(+~1~2)$.  This is an
example substitution, where we substitute the value $2$ for the
variable $x$ in $(\l x.+~1~x)$.  Formally, we define substitution as
follows.

\begin{definition}
The substitution of a term, $t'$ for a variable $x \in V$ in a term
$t$, denoted by $[t'/x]t$, is an instance of $t$ where all the free
occurrences $x$ is replaced by the term $t'$.
\end{definition}

Throughout this course, we will use the notation $[t'/x]t$ to denote a
substitution of $t'$ for $x$ in $t$.  Different notations are
preferred by different textbooks or authors. Two other commonly used
notations are $t[t'/x]$ and $[x:=t']t$.  Pierce's book uses the
notation $[x \rightarrow t']t$.


\begin{example}
Some example substitutions.
\begin{enumerate}
\item $[\l y.y/x] (x~x)= (\l y.y)~(\l y.y)$

\item $[\l y.y/x] (\l x.x) = \l x.x$ ($x$ is not free.)

\item $[y/x](\l z.x) = \l z.y$

\end{enumerate}
\end{example}

A substitution, as we defined it, is actually ``incorrect''.  For
example, consider the lambda abstraction $\l y.x$, the constant
function, and the substitution $[y/x](\l y.x)$, which is equal to $\l
y.y$, the identity function.  Thus, the substitution would naively
change the constant function into the identity function.  Here, the
problem is that the variable $x$, a free variable, is substituted with
$y$, which then becomes bound by the lambda abstraction.  In this
case, we say that $y$ is \emph{captured} in the substitution.

We do not want to transform free variables into bound variables during
the substitution process.  We need to redefine substitution to avoid
capture. What can we do?  Consider the following situation: given a
program which has a function that takes a variable $x$ and has a
variable $y$ in its body, we can change the name of $y$ to $z$ without
any modification of the results, but we cannot change the name of $y$
to $x$ without adversely affecting the results. The next definition
avoids capture.
 
\begin{definition} [Capture Avoiding substitution]
\label{def:sub}
\[
[t/x]y = \left\{ \begin{array}{ll}
        t & if~y = x \\
        y & if~y\neq x \\
        \end{array} \right.
\]


\[
[t/x](t_1~t_2) = [t/x]t_1~[t/x]t_2.
 \]

\[
[t'/x](\l y.t) = 
\left\{ \begin{array}{ll}

\l y.t &\mbox{if}~ x = y \mbox{ (y is bound)} \\

\l y.[t'/x]t &\mbox{if}~ x \neq y \mbox { and } y \not \in FV(t') \\

\end{array} \right. 
\]
\end{definition}

\begin{flex}
\begin{exercise}
What does the substitution $[\l y.x~y~y/z](\l x.x~z)$ yield?
\end{exercise}
\begin{solution}
This substitution is undefined.
\end{solution}
\end{flex}

\begin{flex}
\begin{exercise}
How about the substitution  $[\l y.x~y~y/z](\l y.y~z)$
\end{exercise}
\begin{solution}
$\l y.y~(\l y.x~y~y)$
\end{solution}
\end{flex}

According to this definition a substitution that causes capture is
undefined.  But something is just not right, because the lambda
abstractions $\l x.x~z$ and $\l y.y~z$ differ only in the name of
their bound variables.  Lambda abstractions denote functions, thus
these two lambda abstractions are the same; it should not matter what
the bound variables are named.

\section{Alpha Conversion and Alpha Equivalence}
One way to avoid capture is to rename the bound variables in a lambda
abstraction.  To goal is to ensure that no bound variable has the same
name as a free variable in the term being substituted.  

This process of renaming the bound variables is called
$\alpha$-conversion or $\alpha$-variation.  Two terms that are
reducible to each other by $\alpha$-conversions are \emph{alpha
  equivalent}. We denote $\alpha$-equivalent terms $t_1$ and $t_2$ as
$t_1 \eqa t_2$.  

\begin{enumerate}
\item $\l x.x \eqa \l y.y$. 
\item $\l x. \l y. \l z. x ~y~z \eqa \l z. \l y. \l x. z~y~x$.
\item $\l x.t \eqa \l y.[y/x]t$ if  $y \not\in FV(t)$.
\end{enumerate}

\begin{exercise}
Show that $\eqa$ is an equivalence relation.
\end{exercise}


Alpha conversion gives us a way to define substitution without
worrying about capture.  
\begin{definition}[Substitution with Explicit Alpha Conversion]
\[
[t/x]y = \left\{ \begin{array}{ll}
        t & if~y = x \\
        y & if~y\neq x \\
        \end{array} \right.
\]


\[
[t/x](t_1~t_2) = [t/x]t_1~[t/x]t_2.
 \]



\[[t'/x](\l y.t)  =
\left\{ \begin{array}{ll}
        \l y.t&\mbox{if}~x = y \\
        \l z. [t'/x] [z/y] t
        &\mbox{if}~ x \not= y ~ \land ~ z \not\in FV(t) \cup FV(t') \\
       \end{array} \right. \]
\end{definition}

The idea of this definition is to rename formal variable of the lambda
abstraction so that it does not occur freely in $t'$.  This ensures
that no free variables of $t'$ are captured.  This is not enough,
however, because we also have to make sure that we do not capture the
free variables of $t$ itself. 

Note that, in this definition, substitution is a relation not a
function.  That is the result of a substitution is a set of terms
where the renaming take different forms.  More precisely, the variable
$z$ can take many values (the names of difference variables).  All
such terms are $\alpha$-equivalent. 

Now that we made precise the idea of alpha conversion, we can now
forget about it.  From now on, we will work modulo
$\alpha$-equivalence.  That is, we will not distinguish between two
terms that are $\alpha$-equivalent.  We can use our old definition,
with implicit alpha conversion applied as required.


\begin{definition}[Substitution with Implicit Alpha Conversion]
\label{def:sub3}
\[
[t/x]y = \left\{ \begin{array}{ll}
        t & if~y = x \\
        y & if~y\neq x \\
        \end{array} \right.
\]


\[
[t/x](t_1~t_2) = [t/x]t_1~[t/x]t_2.
 \]

\[
[t'/x](\l y.t) = 
\left\{ \begin{array}{ll}

\l y.t &\mbox{if}~ x = y \mbox{ (y is bound)} \\

\l y.[t'/x]t &\mbox{if}~ x \neq y \land y \not \in FV(t') \\

\end{array} \right. 
\]
\end{definition}


\section{$\beta$-reduction}

In \secref{substitution}, we briefly discussed how to evaluate a
function with a given parameter by substitution.  We can give an
operational semantics for lambda terms based on substitution.  The
idea is to take each application and reduce it to another term by
applying substitution---this is called \emph{$\beta$ reduction}, and is
denoted $\redb$.

\begin{definition}[Single-step $\beta$-reduction]
\[
\begin{array}{c}
(\l x.t_1)~t_2 \redb [t_2/x] t_1 \\[4mm]
\infer{t_1~t_2 \redb t_1'~t_2}{t_1 \redb t'_1}\\[4mm]
\infer{t_1~t_2 \redb t_1~t'_2}{t_2 \redb t'_2}\\[4mm]
\infer{\l x.t \redb \l x.t'} {t \redb t'}
\end{array}
\]
\end{definition}


We define multi-step beta reduction, denoted $\redbs$, as $0$ or more
applications of single-step beta reduction rules.  The use of $*$ to
denote reductions suggests their use in Kleene Algebra of Automata
Theory.

\begin{definition}[Multi-step $\beta$-reduction]
\[
\begin{array}{c}
\infer{\strut} {t \redbs t}\\[4mm]
\infer{t \redbs t'} {t \redb t'} \\[4mm]
\infer{t_1 \redbs t_3} {t_1 \redbs t_2 \\t_2 \redbs t_3} \\[4mm]
\end{array}
\]
\end{definition}

Two terms that are $\beta$-reducible to each other are called
$\beta$-equivalent, denoted by $\eqb$. 

\begin{definition}[$\beta$-equivalence]
\[
\begin{array}{c}
\infer{\strut} {t \eqb t} \\[4mm]
\infer{t \eqb t'} {t \redb t'}  \\[4mm]
\infer{t_1 \eqb t_3} {t_1 \eqb t_2 \\t_2 \eqb t_3}  \\[4mm]
\infer{t' \eqb t} {t \eqb t'}
\end{array}
\]
\end{definition}


\subsection{Evaluation Strategies}

The only means of computing the lambda calculus is beta reduction,
i.e., the application of a function to its arguments.  In beta
reduction, the expression being reduced is called a \emph{reducible
  expression or a redex}.  Given a term, we evaluate that term by
applying beta reduction to its redexes.  There a often multiple
redexes in a term, and, in general, we can choose to evaluate any one
of them.  The following \emph{evaluation strategies} have been
proposed.  As we will see in the rest of the class, there are
interesting differences between these strategies in terms of
``run-time'' behavior of programs.

\begin{description}
\item[Full beta reduction:] Any redex can be reduced at any time.
\item[Normal order strategy:] The leftmost and outermost redex is
  reduced next.
\item[Call by name strategy:] The leftmost and outermost redex is
  reduced next but no reductions are allowed under abstractions.
\item[Call by value:] Like call by name but a redex is reduced only
  when its right hand-side is a value.
\end{description}


Here is a call-by-value operational semantics for lambda calculus. 

\[
\begin{array}{c}
\infer{t_1~t_1 \red t_1'~t_2}{t_1 \red t_1'}\\[2mm]
\infer{v_1~t_2 \red v_1~t_2'}{t_2 \red t_2'}\\
\infer{(\l x.t) v \red [v/x]~t}{\strut}
\end{array}
\]

\section{Homework Exercise}
A term $t$ is in normal form if there is no $t'$ such that $t \redb
t'$.  We say that a term $t$ is \emph{normalizable} if there is some
$t'$ such that $t \redbs t'$ and $t'$ is in normal form.  Answer the
following questions and prove your answer (if your answer is yes, an
example suffices).

\begin{enumerate}
\item Are there any normalizable terms?
\item Are there any non-normalizable terms?
\end{enumerate}







