\chapter{Lambda Calculus: History and Syntax}
\label{ch:lcs}


\section{A Brief History}
\label{sec:lcs::history}

\begin{gram}
\label{grm:lcs::history}
In 1928, Hilbert and Ackerman posed a challenge: devise an algorithm that takes as input a  first-order logic statement and determines whether that statement is valid or not.
%
Soon after, Alonzo Church, then Professor at the Department of Mathematics in Princeton, 
started working on this problem.  His approach was to research the notion of ``function'' and create based on this notion a logical system that is sufficient for all of mathematics.
%
Lambda calculus emerged out of this research, also with contributions from 
Church's students Kleene and Rosser.
%
This research led to Church's 1936 paper 
\href{http://www.umut-acar.org/greats/church-1936.pdf}{showing that an algorithm as desired Hilbert and Ackerman's does not exist}.
%
His solution was to formulate a term in Lambda Calculus and show that there is no way to determine whether that term has a closed form (more precisely $\beta$-normal form). 
%
About one year later, Turing published \href{http://www.umut-acar.org/greats/turing-1937.pdf}{his paper}, where he establish the same result but using different techniques that are based on ``computing machines'', and 
%
proved that his and Church's approach were equivalent.

\end{gram}

\begin{gram}[Church and Turing]
\label{grm:lcs::church-and-turing}
Church and Turing's results are like two sides of a coin.  
%
Church's result is all about abstraction  offers a mathematical language in which computation can be expressed.
%
Turing's result is all about implementation: it convincingly describes how to implement computation.
\end{gram}

\section{Abstract Syntax of Lambda Calculus}
\label{sec:lcs::syn} 

There are at least several ways to define the syntax of Lambda Calculus.
%
In this section, we go through these different ways, partly because they introduce some basic techniques that we shall use in this course.  

\begin{definition}[Inductive Definition]
\label{def:lcs::syn::inductive}
Let $V$ be a countable set $V$ of variables.  We define the abstract
syntax for lambda calculus inductively as follows.
%
$\terms$ is the {\em least} set of the terms that satisfy the following.
\begin{enumerate}
\item{if $x\in V$ then $x\in {\cal T}$}
\item{if $t_1\in {\cal T}$ and $t_2\in {\cal T}$ then $t_1t_2\in {\cal T}$}
\item{if $x\in V$ and $t\in {\cal T}$ then $\l   x.t\in {\cal T}$}
\item{${\cal T}$ is the ``least'' set verifying the above properties}
\end{enumerate}

Each term in $\terms$ is called a \defn{lambda term}.  
\end{definition}

\begin{example}
\label{xrcs:lcs::syn::1} 
Example lambda terms include
\begin{itemize}
\item $x$,
\item $\l x.x$, and
\item $\l x.x~y$.  
\end{itemize}
\end{example}

\begin{definition}[Lambda Abstraction and Application]
\label{def:lcs::syn::abstraction}
The term $\l x.x$ is called \defn{(lambda) abstraction},
%
and the term $t_1~t_2$ is known as \defn{ application}.
%

An intuitive way of thinking of $\l x.t$ is as a function that takes
$x$ and computes the result in its body $t$.
\end{definition}

\begin{gram}[Abstract versus Concrete Syntax]
\label{grm:lcs::syn::abs-and-conc} 
This inductive definition of lambda calculus is an \emph{abstract syntax}:
%
it defines the set of properly parsed terms (i.e., abstract syntax
trees).  
%

It does not tell us how to read a lambda term as written in
\emph{concrete syntax}.  
%
For example, given $\l x. x~y$, we can parse it as 
$(\l x.x~y)$ or $(\l x. x) y$.  
%
Similarly, we can parse $t_1~t_2~t_3$ as
$(t_1~t_2)~t_3$ or $t_1~(t_2~t_3)$.
\end{gram}

\begin{gram}[Disambiguation Conventions]
\label{grm:lcs::syn::conventions} 
We will use parenthesis to aid in parsing (to disambiguate the
syntax).  To minimize parenthesis, we will have the following
conventions:
\begin{enumerate}
\item Application associates to the left.
\item The body of a lambda terms extends as far as right as possible. 
\end{enumerate}

With this convention $\l x. x~y$ is parsed as $\l x. (x~y)$ and $t_1~
t_2~t_3$ is parsed as $(t_1~t_2)~t_3$.
\end{gram}


\begin{definition}[Inference Rules]
\label{def:lcs::syn::inference}
We can define the abstract syntax of Lambda Calculus by using inference rules.
%
Given a countable set of variables $V$, the set of lambda terms is
defined as follows.

\begin{enumerate}

\item{$\infer {x\in {\cal T}} {x\in V}$}

\item{$\infer  {t_1~t_2\in {\cal T}} {t_1\in {\cal T} \\t_2\in {\cal T}}$}

\item{$\infer  {\l x.t\in {\cal T}} {x\in V \\t\in {\cal T}}$}
\end{enumerate}

\end{definition}

\begin{exercise}
Convince yourself that the two definitions above, 
%
the  \href{def:lcs::syn::inductive} and \href{def:lcs::syn::inference}
%
are equivalent.
\end{exercise}

\begin{definition}[BNF Style]
\label{def:lcs::syn::bnf} 
We can define the syntax of Lambda Calculus by using the BNF style. 
%
Assuming that $x$ ranges over a countable set of variables, the set of
lambda terms $t$ is defined as follows.

\[
t \bnfdef x \bnfalt t_1 t_2 \bnfalt \l x.t
\]
\end{definition}



\begin{exercise}
\label{xrcs:lcs::syn::bnf} 
Suppose that our syntax allow us to write natural numbers and add
them.  For example, we may have terms like this $1, 2, +~1~2$.

Now, what does the following lambda terms do?
\begin{enumerate}
\item $\l x. +~x~1$,
\item $\l x. \l y. +~x~y$,
\item $\l x. \l y. \l z.  z (+~x~y)$.
\end{enumerate}
\end{exercise}


\begin{gram}[Summary]
\label{grm:lcs::syn::summary} 
What is remarkable about Lambda Calculus is that just a one line definition suffices to define all of computation.
%
We don't need to talk about ``tapes'', ``cells'' ``instructions'', ``states'', etc, which are needed even in the most basic definition of Turing Machine.
%
Not only it is  elegant, but it is also powerful: it allows expressing sophisticated algorithms clearly and concisely.
%

\end{gram}

\begin{note}
\label{nt:lcs::syn::pl} 
Even many modern programming languages today lack the features of Lambda Calculus.
%
Though many are also busy trying to add them (which is not always easy).
%
For example,
the C++17 has now support for Lambda expressions as outlined 
%
\href{http://www.open-std.org/jtc1/sc22/wg21/docs/papers/2016/p0170r1.pdf}
{in this document}
%
from the C++ Standard Committee.

For good measure, perhaps they could have waited a few more
years to match the 100th year invention of Lambda Calculus!
\end{note}


\section{Bound and Free Variables}
\label{sec:lcs::variables}

\begin{definition}[Parameters and Binding]
\label{def:lcs::binding}
Consider a lambda abstraction of the form $\l x. t$, which denotes a
function.
%
We refer to the variable $x$ as the \defn{formal parameter} or more
simply as the \defn{parameter} of the abstraction (function),
%
and
%
say that the abstractions \defn{binds} $x$.
%
\end{definition}

\begin{flex}
\begin{definition}[Bound and Free Variables]
\label{def:lcs::bound-and-free}
The occurrence of a variable is \defn{bound} if there is an enclosing
lambda abstraction that binds the variable, and is \defn{free}
otherwise.
\end{definition}

\begin{example}
Consider the lambda abstraction, $\l x. x + y$.  In order to evaluate
the function, for a particular argument $x$, we need to know
the value of $y$.  The variable $y$ in this case is free and $x$
is bound; $\l$ binds $x$.  
\end{example}


\begin{example} 
  Some example lambda abstractions and their bound and free variables.
\begin{enumerate}
\item $\l x. x$ is the identity function. Here $x$ is a bound variable.
\item $\l x. y$ is the constant ``$y$'' function. Here $y$ is a free variable.
\item $\l x. x~y$ has $x$ as a bound variable and $y$ as a free variable.
\item $\l x. (\l y. x~y)$ has both variables bound.
\item $~x~\l x. x + 1$.  In this term the first occurrence of $x$ is
  free and the second is bound.
\end{enumerate}
\end{example}
\end{flex}

\begin{definition}[Free Variables]
\label{def:lcs::fv}
The set of \defn{free variables} of a term $t$, written as $FV(t)$,
is defined as 
\begin{enumerate}
\item if $t = x$ then $FV(x) = \{ x \}$, or
\item if $t = t_1~t_2$ then $FV(t_1~t_2)=FV(t_1) \cup FV(t_2)$, and
\item if $t = \l x. t$ then $FV(\l x.t)=FV(t) \setminus \{x\}$
\end{enumerate}
\end{definition}

\begin{definition}[Closed and Open Terms]
\label{def:lcs::open-closed}
A term $t$ is closed if $FV(t) = \varnothing$. Otherwise $t$ is open. 
\end{definition}


\section{Alpha Conversion and Alpha Equivalence}
\label{sec:lcs::alpha}
In any abstraction, the formal parameter acts simply as a placeholder and therefore we can change as long as we don't create a ``clash'' with a free variable.
%
For example, consider the abstraction $\l x.t$, where $x$ is the argument. We can rename $x$ to any variable, say $y$, as long as $y$ is not a free variable in $t$.
%

\begin{flex}
\begin{definition}[Alpha Conversion and Alpha Equivalence]
\label{def:lcs::alpha}
An \defn{$\alpha$-conversion} or \defn{$\alpha$-variation} of a lambda
term $t$ is
%
another term $t'$ where the a bound variable $x$ of $t$ is renamed to another variable $y \not\in FV(t)$.
%
We say than two terms that are reducible to each other by $\alpha$-conversions are \defn{alpha equivalent}
%
and
%
denote $\alpha$-equivalent terms $t_1$ and
$t_2$ as $t_1 \eqa t_2$.
\end{definition}

\begin{example}
\begin{enumerate}
\item $\l x.x \eqa \l y.y$. 
\item $\l x.y \neqa \l y.y$. 
\item $\l x. \l y. \l z. x ~y~z \eqa \l z. \l y. \l x. z~y~x$.
\item $\l x.t \eqa \l y.[y/x]t$ if  $y \not\in FV(t)$.
\end{enumerate}
\end{example}
\end{flex}

\begin{exercise}
\label{xrcs::alphha-eq-is-eq}
Show that $\eqa$ is an equivalence relation.
\end{exercise}


